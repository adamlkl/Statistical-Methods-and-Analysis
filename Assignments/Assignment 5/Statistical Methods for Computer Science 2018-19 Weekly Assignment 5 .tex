\documentclass[12pt]{article}%
\usepackage{amsfonts}
\usepackage{fancyhdr}
\usepackage{comment}
\usepackage[a4paper, top=2.5cm, bottom=2.5cm, left=2.2cm, right=2.2cm]%
{geometry}
\usepackage{times}
\usepackage{amsmath, nccmath}
\usepackage{changepage}
\usepackage{amssymb}
\usepackage{graphicx}%
\usepackage{lipsum}
\usepackage{array}
\usepackage{listings}
\usepackage{color}
\usepackage{xcolor}

\delimitershortfall-1sp
\newcommand\abs[1]{\left|#1\right|}
\graphicspath{ {./images/} }

\definecolor{lightgray}{RGB}{214, 219, 223}
\definecolor{limegreen}{RGB}{11, 83, 69}
\definecolor{blue}{RGB}{0, 70, 255}
\lstdefinestyle{mystyle}{
    backgroundcolor=\color{lightgray},   
    commentstyle=\color{limegreen},
    keywordstyle=\color{blue}
}
 
\lstset{style=mystyle}
\makeatletter
\renewcommand{\maketitle}{\bgroup\setlength{\parindent}{0pt}
\begin{flushleft}
  \textbf{\@title}

  \@author
  \@date
\end{flushleft}\egroup
}
\makeatother


\begin{document}

\title{Weekly Assignment 5}
\author{Leong Kai Ler \\ 15334636 \\   }
\date{March 5, 2019}
\maketitle

\section*{Question 1}
A box contains 5 red and 5 blue marbles. Two marbles are withdrawn randomly. If they are the same color, then you win \$$1.10$; if they are different colors, then you lose \$$1.00$. Calculate:
\subsection*{Problem Part a:}
The expected value of the amount you win.
\subsection*{Solution Part a:}
There are two possibilities that satisfy the event, either we draw two red marbles or two blue ones. Let say we want to draw two red marbles, there are $\frac{5}{10}$ chance to get a red marble in first draw and $\frac{4}{9}$ of getting a second one. Same calculations apply for the getting two blue marbles. Hence, probability of getting two marbles of same color is $\frac{5}{10}*\frac{4}{9}*2=\frac{4}{9}$.
\begin{eqnarray*}
E[X] & = & P(E) * 1.1 + (1-P(E))*(-1.0) \\
     & = & \frac{4}{9} * 1.1 + \frac{5}{9} * (-1.0) \\
     & = & -\frac{1}{15} \\
     & \approx & -0.0667
\end{eqnarray*}
\subsection*{Problem Part b:}
The variance of the amount you win.
\subsection*{Solution Part b:}
\begin{eqnarray*}
E[X^2] & = & \frac{4}{9} * 1.1^2 + \frac{5}{9} * (-1.0)^2 \\
	   & = & \frac{82}{75}\\
Var[X] & = & E[X^2] - (E[X])^2 \\
	   & = & \frac{82}{75} - (-\frac{1}{15})^2 \\
	   & = & \frac{49}{45} \\
	   & \approx & 1.08889
\end{eqnarray*}

\newpage
\section*{Question 2}
Suppose you carry out a poll following an election. You do this by selecting n people uniformly at random and asking whether they voted or not, letting Xi = 1 if person i voted and Xi = 0 otherwise. Suppose the probability that a person voted is 0.6.
\subsection*{Problem Part a:}
Calculate $E[X_i]$ and $Var(X_i)$. \\ 
\subsection*{Solution Part a:}

\subsection*{Problem Part b:}
Let $Y = \sum_{i=1}^{n} X_i$. \\
What is $E[Y]$ ? Is it the same as $E[X]$ or different, and why ?
\subsection*{Solution Part b:}

\subsection*{Problem Part c:}
What is $E[\frac{1}{n}Y]$ ?
\subsection*{Solution Part c:}

\subsection*{Problem Part d:}
What is the variance of $\frac{1}{n}Y$ (express in terms of $Var(X)$)? \\
Hints: use linearity of the expectation and the fact that people are sampled independently.
\subsection*{Solution Part d:}

\newpage
\section*{Question 3}
Suppose that 2 balls are chosen without replacement from an urn consisting of 5 white and 8 red balls. Let $X_i$ equal 1 if the $i^{th}$ ball selected is white, and let it equal 0 otherwise.
\subsection*{Problem Part a:}
Give the joint probability mass function of $X_1$ and $X_2$
\subsection*{Solution Part a:}
To calculate joint probability mass function of $X_1$ and $X_2$, we need to compute their distributions.
\begin{itemize}
\item 
\begin{fleqn}[\parindent]
\begin{equation*}
\begin{split}
P(X_1 = 0, X_2 = 0) = & \frac{8}{13}*\frac{7}{12}\\
					= & \frac{14}{39}
\end{split}
\end{equation*}
\end{fleqn}
\item 
\begin{fleqn}[\parindent]
\begin{equation*}
\begin{split}
P(X_1 = 1, X_2 = 0) = & \frac{5}{13}*\frac{8}{12}\\
					= & \frac{10}{39}
\end{split}
\end{equation*}
\end{fleqn}
\item 
\begin{fleqn}[\parindent]
\begin{equation*}
\begin{split}
P(X_1 = 0, X_2 = 1) = & \frac{8}{13}*\frac{5}{12}\\
					= & \frac{10}{39}
\end{split}
\end{equation*}
\end{fleqn}
\item 
\begin{fleqn}[\parindent]
\begin{equation*}
\begin{split}
P(X_1 = 1, X_2 = 1) = & \frac{5}{13}*\frac{4}{12}\\
					= & \frac{5}{39}
\end{split}
\end{equation*}
\end{fleqn}
\end{itemize}
Hence, the tabular form of joint pmf would look as illustrated below: \\
\renewcommand{\arraystretch}{2}
\begin{center}
\begin{tabular}[5pt]{| r | c | c | c |}
\hline
 & $X_1 = 0$ & $X_1 = 1$ & $P(X_1 = x_1)$ \\
\hline
$X_2 = 0$ & $\frac{14}{39}$ & $\frac{10}{39}$ & $\frac{24}{39}$\\
\hline 
$X_2 = 1$ & $\frac{10}{39}$ & $\frac{5}{39}$ & $\frac{15}{39}$\\
\hline
$P(X_2 = x_2)$ & $\frac{24}{39}$ & $\frac{15}{39}$ & $1$ \\
\hline 
\end{tabular}
\end{center}
\renewcommand{\arraystretch}{1}
\subsection*{Problem Part b:}
Are $X_1$ and $X_2$ independent ? (Use the formal definition of independence to determine this)
\subsection*{Solution Part b:}
For $X_1$ and $X_2$ to be independent, we need to prove: \begin{equation*}
P(X_1=x_1 \cap X_2=x_2) = P(X_1=x_1) * P(X_2=x_2)
\end{equation*}
for all $x_1 \in R_{X_1}$ and $x_2 \in R_{X_2}$. \\ \\
Take an example of $P(X_2=1)$ and $P(X_2=1 | X_1=1)$:
\begin{eqnarray*}
P(X_2=1) & = & \frac{15}{39} \\
P(X_2=1 | X_1=0) & = & \frac{P(X_1=0 | X_2=1)}{P(X_1=0} \\
				 & = & \frac{\frac{10}{39}}{\frac{24}{39}} \\
				 & = &\frac{5}{12} 
\end{eqnarray*}
Since $P(X_2=1)\neq P(X_2=1 | X_1=0)$, $X_1$ and $X_2$ are not independent. 
\subsection*{Problem Part c:}
Calculate $E[X_2]$
\subsection*{Solution Part c:}
\begin{eqnarray*}
E[X_2] & = & \frac{15}{39} * 1 + \frac{24}{39} * 0 \\
	   & = & \frac{5}{13} \\
	   & \approx & 0.3846
\end{eqnarray*}
\subsection*{Problem Part d:}
Calculate $E[X_2|X1 = 1]$
\subsection*{Solution Part d:}
\begin{eqnarray*}
E[X_2|X1 = 1] & = & \frac{P(X_2 = x_2)}{P(X_1=1)} \\
	   & = & \frac{\frac{5}{39}*1+\frac{10}{39}*0}{\frac{15}{39}} \\
	   & = & \frac{1}{3} \\
	   & \approx & 0.3333
\end{eqnarray*}
\end{document}